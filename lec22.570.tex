\documentclass[10pt]{article}
%%%
%\usepackage{latexsym}
\usepackage{epsfig}
\usepackage{amssymb}

\begin{document}
\textwidth=8in
\textheight=9in
\parindent=10pt
\hoffset=-1in
\voffset=-.5in
\parskip=.065in
\newtheorem{problem}{Problem}

\begin{center}
\begin{tabular}{|lcr|}
\hline
.\hspace{1in}$$&$$\hspace{2in}$$&$$\hspace{1in}. \\
{\large \textsf{Fall 2011}} & 
{\large \textsf{\textbf{CSC 570:} Bioinformatics}} &
{\large \textsf{ Alexander Dekhtyar}}\\
.\hspace{1in}$$&$$\hspace{2.5in}$$&$$\hspace{1in}. \\
\hline	
\end{tabular}
\end{center}

\begin{center}
\textsf{\large Approximate String Matching}
\end{center}

{\large \textbf{Prepared by:} \textit{Ryan Schmitt, Bob Somers, \& Jim Langlois.}
}

\section*{Section Header}
%%%
%%%  use \section*{}, \subsection*{}, \subsubsection*{}  (with *)
%%%

\paragraph{File names.} Name your file according to the instructions on the
web site. The template for the .tex file name is \texttt{lecNN.570.tex}. The \texttt{NN}
number is assigned to each team on the web page.

\paragraph{Problem Specification.} Use $\backslash$\texttt{paragraph} headings for
labeled paragraphs. These include paragraphs which need titles, as well
as examples, theorems etc.:

\paragraph{Example.} This is an example.

\paragraph{Theorem.} This is a theorem statement.

\paragraph{Formulas.} Use \$\$ \ldots \$\$ environment for long-ish formulas:

$$ sim(\bar{x},\bar{y}) = \frac{\sum_{i=1}^{n} x_i\cdot y_i}{\sqrt{\sum_{i=1}^{n}
x_i^2}\cdot \sqrt{\sum_{i=1}^{n}y_i^2}}.$$


\subsection*{Figures}

 Figures are best drawn in \texttt{xfig}. They can be saved in \texttt{.fig} files
and exported into Encapsulated Postscript, \texttt{.eps} format. And example is
shown in Figure \ref{fig:lca}. Pick an appropriate width for your images, and
center them.

\begin{figure}
\begin{center}
\psfig{file=lca.eps, width=3in}
\end{center}
\caption{Illustrating the definition of lowest common ancestor: node $z$ is
the \textit{lca} for nodes $x$ and $y$.}\label{fig:lca}
\end{figure}


\subsection*{Algorithms}

Use the setup shown below to create the pseudocode for the algorithms.
It is not the most elegant way to create algorithm descriptions, but the result
renders pretty well.


\fbox{
\begin{minipage}{8cm}
\textsc{Algorithm} \textsf{NaiveStringMatch(S,P,n,m)}\\
\textbf{begin}\\
$\mbox{ }\mbox{ }$ \textbf{for} \textsf{k}$=1$ \textbf{to} $n-m+1 \mbox{ }\mbox{ }$ \textbf{do}\\
$\mbox{ }\mbox{ }\mbox{ }\mbox{ }$ \textbf{if} $P = S[k,k+m-1] \mbox{ }\mbox{ }$ \textbf{then}\\
$\mbox{ }\mbox{ }\mbox{ }\mbox{ }\mbox{ }\mbox{ }$ \textbf{output}(k);\\
$\mbox{ }\mbox{ }$\textbf{end for}\\
\textbf{end}\\
\end{minipage}
}



\subsection*{Bibiliography}

 Bibiliography should accompany all your submissions. The simplest way is the
\texttt{thebibiliography} environment, as shown in this file.


\begin{thebibliography}{999}

\bibitem{gusfield} Dan Gusfield, \textit{Algorithms on Strings, Trees and Sequences: Computer Science
and Computational Biology},  Cambridge University Press, 1997.

\end{thebibliography}



\end{document}
