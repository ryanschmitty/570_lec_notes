\documentclass[10pt]{article}
%%%
%\usepackage{latexsym}
\usepackage{epsfig}
\usepackage{amssymb}

\begin{document}
\textwidth=8in
\textheight=9in
\parindent=10pt
\hoffset=-1in
\voffset=-.5in
\parskip=.065in
\newtheorem{problem}{Problem}

\begin{center}
\begin{tabular}{|lcr|}
\hline
.\hspace{1in}$$&$$\hspace{2in}$$&$$\hspace{1in}. \\
{\large \textsf{Fall 2011}} & 
{\large \textsf{\textbf{CSC 570:} Bioinformatics}} &
{\large \textsf{ Alexander Dekhtyar}}\\
.\hspace{1in}$$&$$\hspace{2.5in}$$&$$\hspace{1in}. \\
\hline	
\end{tabular}
\end{center}

\begin{center}
\textsf{\large Approximate String Matching}
\end{center}

{\large \textbf{Prepared by:} \textit{Ryan Schmitt, Bob Somers, \& Jim Langlois.}
}


%%
%%  RYAN'S TEXT SECTION BEGINS HERE
%%

\section*{Approximate String Matching and Filtering}

\paragraph{}Filtering is another method for approximate string matching. Below is a simple overview of how filtering algorithms are designed:

\begin{list}{}{}
\item 1) Filtering Phase
\begin{list}{*}{}
\item input: source text, patternjjwkkk
\item algorithm quickly scans source for potential pattern matches
\item output: list of potential matches
\end{list}
\item 2) Verification Phase
\begin{list}{*}{}
\item input: list of potential matches
\item algorithm slowly verifies potential matches as exact matches
\item output: list of exact matches
\end{list}
\end{list}

The goal here is to mask out sections of the text that cannot contain matches, and only use the expensive algorithm on sections that could have matches.
For an overview of the twelve filtering algorithms please see \cite{navarro99}.

For specific algorithms see the following list:

\begin{list}{*}{}
\item Faster approximate string matching \cite{baeza99}
\item Sublinear approximate string matching and biological applications \cite{chang94a}
\item Approximate string matching and local similarity \cite{chang94b}
\item A comparison of approximate string matching algorithms \cite{jokinen96}
\item A Guided Tour to Approximate String Matching \cite{navarro99}
\item Improving an algorithm for approximate pattern matching \cite{navarro98}
\item Fast and flexible string matching by combining bit-parallelism and suffix automata \cite{navarro00}
\item Fast approximate string matching with q-blocks sequences \cite{shi96}
\item On using q-gram locations in approximate string matching \cite{sutinen95}
\item Approximate pattern matching with samples \cite{takaoka94}
\item Approximate Boyer-Moore string matching \cite{taruk93}
\item Approximate string matching with q-grams and maximal matches \cite{ukkonen92}
\item Fast text searching allowing errors \cite{wu92}
\end{list}

One of the forms of filtering is based on the Boyer-Moore algorithm that we have
studied previously (specifically the Boyer-Moore-Horspool (BMH) variation, where
the match heuristic is ignored, and only a simplified occurrence heuristic is used
\cite{taruk93}). And the genesis of this type of filtering is a 1993 paper by Tarhio
and Ukkonen, where they extend the BMH algorithm for the k-differences problem \cite{taruk93}.

For a review of the Boyer-Moore algorithm, please see the String Matching lecture notes.

\subsection*{The K-Differences Problem}

\paragraph*{Problem Statement}Given a text, T, and a pattern, P, find all occurrences of P
in T with at most k mismatches \cite{taruk93}. This is related to the Hamming Distance.

\subsubsection*{Example}

\noindent T = ``emample example''

\noindent P = ``example''

The k-differences problem with k=0 has only one solution, where ``example'' matches ``example.''
However, with k=1, there are two occurrences of ``example'' in ``emample example'' with at most 1 mismatch.

\subsection*{Approximate BMH Algorithm}

This extension to the Boyer-Moore-Horspool algorithm solves the k-differences problem, and, while not a
filtering algorithm itself, was the basis for Tarhio and Ukkonen to develop one of the first filtering algorithms
\cite{taruk93}. The crux of this extension is the generalization of both the right-to-left scanning of the pattern
against the text, and the shift computation.

The naming conventions we will use are as follows: for a text, T, of length n, the position index is h, and for a
pattern, P, of length m, the position index is i. We will find all occurrences of P in T with at most k mismatches.

\paragraph{Right-to-Left Scanning}
Instead of scanning until one mismatch is found, we scan until $k+1$ mismatches are found.

For example, if our pattern is ``abbb'' our text is ``abaacbb'' we have this as our first alignment:

\vspace{12pt}

\noindent \verb/a b b b/

\noindent \verb/a b a a c b b/

\vspace{12pt}

If k=1, we know it's not a match after we’ve found two mismatches:

\vspace{12pt}

\noindent \verb/a b B B/

\noindent \verb/a b A A c b b/

\vspace{12pt}

However, if k=2, we have a match, since we only found two mismatches.

\paragraph{Shift Computation}
Instead of only basing the shift on the text character paired with the last pattern character, we base the shift on the
last $k+1$ characters. The maximum shift is $m-k$. The shift is computed as the minimum shift out of those defined by the
$k+1$ text characters that are paired with the $k+1$ last characters of the pattern.

For example, if our pattern is ``abbb'' our text is ``abaacbb'' and $k=1$, we have the following as our first alignment:

\vspace{12pt}

\noindent \verb/a b b b/

\noindent \verb/a b a a c b b/

\vspace{12pt}

Here we have greater than k mismatches (2 mismatches) so we don't have a match. We look at the text characters below the
last $k+1$ characters of the pattern to determine our shift. Since $k = 1$, then $k+1 = 2$, and the text characters below the
last 2 characters of the pattern are `a' and `a':

\vspace{12pt}

\noindent \verb/a b B B/

\noindent \verb/a b A A c b b/

\vspace{12pt}

The `a' located at T$_{4}$ prescribes a shift of 3, in order to pair with the `a' at P$_{1}$, and the `a' located at T$_{3}$ prescribes
a shift of 2 to pair with the `a' at P$_{1}$. Because we minimize over all the shifts ($k+1$ of them), we choose the shift of
2, which leaves us with this:

\vspace{12pt}

\noindent \verb/    a b b b/

\noindent \verb/a b a a c b b/

\vspace{12pt}

Again, we have two mismatches (greater than the allowed 1 mismatch), so we don't have a match. We look at the text
characters below the last $k+1$ pattern characters:

\vspace{12pt}

\noindent \verb/    a b B B/

\noindent \verb/a b a a C B b/

\vspace{12pt}

The `b' located at T$_{6}$ prescribes a shift of 1, in order to pair with the `b' at P$_{3}$, and the `c' located at T$_{5}$ prescribes
a shift of 3, because `c' does not occur in the pattern so we can shift past it. We chose the minimum of the prescribed
shifts, which means we choose the shift of 1.

\vspace{12pt}

\noindent \verb/      a b b b/

\noindent \verb/a b a a c b b/

\vspace{12pt}

Here we only have one mismatch, which is not greater than our k=1. Therefore we have a match!

\vspace{12pt}

The full algorithms are as follows:

\fbox{
\begin{minipage}{7.8cm}
\textsc{Algorithm 2} \textsf{Computation of the shift table. \cite{taruk93}}\\
\textbf{begin}\\
$\mbox{ }\mbox{ }$ \textbf{for} \textsf{a in} $\Sigma$ \textbf{do} \textsf{ready[a] := m + 1;}\\
$\mbox{ }\mbox{ }$ \textbf{for} \textsf{a in} $\Sigma$ \textbf{do}\\
$\mbox{ }\mbox{ }\mbox{ }\mbox{ }$ \textbf{for} \textsf{i := m downto m - k} \textbf{do}\\
$\mbox{ }\mbox{ }\mbox{ }\mbox{ }\mbox{ }\mbox{ }$ \textsf{shiftTable[i, a] := m;}\\
$\mbox{ }\mbox{ }$ \textbf{for} \textsf{i := m - 1 downto 1} \textbf{do begin}\\
$\mbox{ }\mbox{ }\mbox{ }\mbox{ }$ \textbf{for} \textsf{j := ready[p}$_{\textsf{i}}$\textsf{] - 1 downto max(i, m - k)} \textbf{do}\\
$\mbox{ }\mbox{ }\mbox{ }\mbox{ }\mbox{ }\mbox{ }$ \textsf{shiftTable[j, p}$_{\textsf{i}}$\textsf{] := j - i;}\\
$\mbox{ }\mbox{ }\mbox{ }\mbox{ }$ \textsf{ready[p}$_{\textsf{i}}$\textsf{] := max(i, m - k)} \textbf{end}\\ 
\textbf{end}\\
\end{minipage}
}

\fbox{
\begin{minipage}{10.5cm}
\textsc{Algorithm 2} \textsf{Approximate string matching with k mismatches. \cite{taruk93}}\\
\textbf{begin}\\
$\mbox{ }\mbox{ }$ \textsf{compute shiftTable from P with Algorithm 1;}\\
$\mbox{ }\mbox{ }$ \textsf{j := m;}\\
$\mbox{ }\mbox{ }$ \textbf{while} \textsf{j} $\le$ \textsf{n + k} \textbf{do begin}\\
$\mbox{ }\mbox{ }\mbox{ }\mbox{ }\mbox{ }\mbox{ }$ \textsf{h := j; i := m; neq := 0;}\\
$\mbox{ }\mbox{ }\mbox{ }\mbox{ }\mbox{ }\mbox{ }$ \textsf{d := m - k;}\\
$\mbox{ }\mbox{ }\mbox{ }\mbox{ }\mbox{ }\mbox{ }$ \textbf{while} \textsf{i} $>$ \textsf{0 and neq} $\le$ \textsf{k} \textbf{do begin}\\
$\mbox{ }\mbox{ }\mbox{ }\mbox{ }\mbox{ }\mbox{ }\mbox{ }\mbox{ }\mbox{ }\mbox{ }$ \textbf{if} \textsf{i} $\ge$ \textsf{m - k} \textbf{then} \textsf{d := min(d, shiftTable[i, t}$_{\textsf{h}}$\textsf{]);}\\
$\mbox{ }\mbox{ }\mbox{ }\mbox{ }\mbox{ }\mbox{ }\mbox{ }\mbox{ }\mbox{ }\mbox{ }$ \textbf{if} \textsf{t}$_{\textsf{h}} \neq$ \textsf{p}$_{\textsf{i}}$ \textbf{then} \textsf{neq := neq + 1;}\\
$\mbox{ }\mbox{ }\mbox{ }\mbox{ }\mbox{ }\mbox{ }\mbox{ }\mbox{ }\mbox{ }\mbox{ }$ \textsf{i := i - 1; h := h - 1} \textbf{end}\textsf{;}\\
$\mbox{ }\mbox{ }\mbox{ }\mbox{ }\mbox{ }\mbox{ }$ \textbf{if} \textsf{neq} $\le$ \textsf{k} \textbf{then} \textsf{report match at postition j;}\\
$\mbox{ }\mbox{ }\mbox{ }\mbox{ }\mbox{ }\mbox{ }$ \textsf{j := j + d} \textbf{end}\textsf{;}\\
\textbf{end}\\
\end{minipage}
}

\vspace{12pt.}

The preprocessing time to create the shift table is $O(m + kc)$, where $c$ is the size of the alphabet. The storage space
required is $O(kc)$. The running time of the approximate BMH algorithm in the worst case is $O(mn)$, however, in
\cite{taruk93} they prove the expected running time is $O(nk (\frac{k}{c} + \frac{1}{m-k})$ if the lengths of all the shifts in the
shift table are mutually independent. \cite{taruk93}


\subsection*{Filtering and Approximate BMH}

We will only mention the basics of how the ideas in this algorithm are applied to filtering, and leave it to the reader
to research it further in \cite{taruk93}.

The authors' algorithm has two phases that they call scanning and checking \cite{taruk93}, which are analogous to the
filtering and verification phase, respectively, mentioned earlier. The verification phase uses the standard dynamic
programming method for checking whether the edit distance between two strings is within some tolerance. It is the
scanning phase we are interested in, which uses the ideas developed in the approximate BMH algorithm to compute the
shift once an alignment of text and pattern is marked as a potential match or not.

The scanning phase performs two operations iteratively: mark and shift. The mark operation marks the current alignment
of text and pattern as promising or not, prompting a check via dynamic programming if it looks like a potential match.
We will not explore the details of how a particular alignment is deemed promising, which can be reviewed in \cite{taruk93}.
The shift operation uses the exact same heuristic as the approximate BMH algorithm. That is, examining the text
characters under the last $k+1$ characters of the pattern.



%%
%%  RYAN'S TEXT SECTION ENDS HERE
%%


%%
%%  RYAN'S BIBLIOGRAPHY SECTION STARTS HERE
%%

\subsection*{Bibiliography}

\begin{thebibliography}{999}

\bibitem{baeza99} Baeza-Yates, R. and Navarro, G., \textit{Faster approximate string matching}, Algorithmica, 1999.
\bibitem{chang94a} Chang, W. and Lawler, E., \textit{Sublinear approximate string matching and biological applications}, Algorithmica, 1994.
\bibitem{chang94b} Chang, W. and Marr, T., \textit{Approximate string matching and local similarity}, Proceedings of the 5th Annual Symposium on Combinatorial Pattern Matching, 1994.
\bibitem{jokinen96} Jokinen, P., Tarhio, J., and Ukkonen, E., \textit{A comparison of approximate string matching algorithms}, Software Practice Exper., 1996.
\bibitem{navarro99} Navarro, G., \textit{A Guided Tour to Approximate String Matching}, ACM Computing Surveys, 1999.
\bibitem{navarro98} Navarro, G. and Baeza-Yates, R., \textit{Improving an algorithm for approximate pattern matching}, Algorithmica, 1998.
\bibitem{navarro00} Navarro, G. and Raffinot, M., \textit{Fast and flexible string matching by combining bit-parallelism and suffix automata}, ACM J. Exp. Algor. 5, 2000.
\bibitem{shi96} Shi, F., \textit{Fast approximate string matching with q-blocks sequences}, Proceedings of the 3rd Annual South American Workshop on String Processing, 1996.
\bibitem{sutinen95} Sutinen, E. and Tarhio, J., \textit{On using q-gram locations in approximate string matching}, Proceedings of the 3rd Annual European Symposium on Algorithms, 1995.
\bibitem{takaoka94} Takaoka, T., \textit{Approximate pattern matching with samples}, Proceedings of ISAAC '94, 1994.
\bibitem{taruk93} Tarhio, J. and Ukkonen, E., \textit{Approximate Boyer-Moore string matching}, SIAM J. Comput., 1993.
\bibitem{ukkonen92} Ukkonen, E., \textit{Approximate string matching with q-grams and maximal matches}, Theor. Comput. Sci, 1992.
\bibitem{wu92} Wu, S., and Manber, U., \textit{Fast text searching allowing errors}, Commun. ACM, 1992.

\end{thebibliography}

%%
%%  RYAN'S BIBLIOGRAPHY SECTION ENDS HERE
%%



\end{document}
